\chapter{Anomaly Detection}

In data mining, \textit{Anomaly Detection} is the problem of finding patterns in data that do not conform to expected behavior \cite{chandola2009anomaly}. These patterns often denote an underlying different process: an anomaly can be defined an observation which deviates so much from the other observations as to arouse suspicions that it was generated by a different mechanism \cite{hawkins1980identification}.

This task finds extensive use in many domains, such as fraud detection for credit cards, insurance or health care, intrusion detection for cyber security, military surveillance and fault detection in critical systems.

Anomalies are also referred to as outliers, novelties, deviations, exceptions or noise. \textit{Noise removal} and \textit{noise accomodation} are related problems, dealing with the removal of unwanted objects before performing data analysis (removal) and immunizing a statistical model against anomalous observations (accomodation).

\section{Challenges}

Anomaly detection is an hard and complex problem, mainly due to the following challenging factors:

\begin{enumerate}
	\item Defining the normal and anomalous regions and their boundaries;
	\item The notion of \textit{normality} may evolve with time;
	\item Anomalies may be different for different domains; A small fluctuation might be significant in one domain while being a normal in another one;
	\item Availability of labeled data;
	\item Noisy data and anomalies can be hard to discriminate.
\end{enumerate}

To justify the variety of strategies and techniques and show the breadth of problem domain, we will outline the several factors determining an Anomaly Detection problem.

\section{Nature of data}

\section{Type of Anomaly}

\section{Data labels}

\section{Output of Anomaly Detection}

\section{Applications}


Anomaly Detection makes use of tools and concepts from a number of different fields, such as machine learning, data mining, information theory, spectral theory and statistics.