
\begin{vplace}[0.7]
\renewcommand*\abstractname{\huge Abstract}

\begin{abstract}

\fontsize{11}{12}\selectfont\baselineskip=1.2em

High Energy Physics experiments involve large amount of complex systems subject to anomalies and malfunctioning. They generate a lot of high dimensional data that must be monitored and analysed by a Data Quality Monitoring team to validate and deliver certified data for physics analysis.

At the European Organization for Nuclear Research (CERN), in the Large Hadron Collider (LHC), the Compact Moun Solenoid (CMS) experiment generates events at a rate of 40 MHz, each one carrying payloads averaging 1 MB. It is the job of the Trigger System to discard the large majority of this data and retain the most interesting ones.

This crucial phase is sensible to malfunctions of many of the underlying parts, from sub detectors to the trigger algorithms configurations.

Shifters in the CMS Control Room use a Rate Monitoring software to monitor Trigger Rates and spot potential problems.

We proceed to improve and renovate this sofware, exploiting the new features to integrate this data with the Run Registry offline logs, obtaining a dataset of genuine and labeled anomalous runs.

On this dataset, we build an experimental Anomaly Detection framework aiming to provide an ML based tool to support the spotting of anomalies from Trigger Rates.

\normalsize

\end{abstract}

\end{vplace}

\thispagestyle{empty}
\addtocounter{page}{-1}

\pagebreak