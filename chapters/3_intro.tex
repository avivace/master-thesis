\chapter{Introduction}

\section{Motivations}

\paragraph{Monitoring a a particle detector} The CMS detector at the LHC particle accelerator is a complex system which needs fast and reliable monitoring. Quick feedback on each of the subsystems is needed to spot and solve problems or the data taken might not be useful for physics analysis \cite{chep2016wbm}. Experts from different systems need to correlate information to investigate underlying problems.

A centralized monitoring solution exposes real time data, historical information, summaries and reports from a series of different sources:

\begin{itemize}
	\item Luminosity, Collision Rates.
	\item Global Trigger, \textbf{Trigger Rates}.
	\item LHC (Beam currents, losses, status, collimators, real time clock, event signal).
	\item Magnet.
	\item Sub-system specific information stored in relational databases.
	\item DAQ.
	\item DQM.
	\item Experimental running conditions indatabase.
	\item Hardware.
	\item Other non event data.
\end{itemize}

Such system is different from the Data Quality Monitoring services, which look at actual \textit{event} data.

The WBM software covered this role since the commissioning of the CMS experiment (2008), evolving and integrating new services into a growing framework during LHC Run 1 (2010-2013) and Run 2 (2015-2018).

\paragraph{Upgrading the monitoring framework} 

During the second Long Shutdown of the LHC (2018-2021) the CMS detector will be upgraded and many CMS sub-systems will drastically change. WBM started to show its age and problems: it unexpectedly and heavily grew with new features, arriving at a point where services were using vastly different technologies and it became harder and harder to mantain and expand \cite{CMSWBMreview}. It has been decided to deprecate \cite{upgradewbmoms} WBM in favor of a new software framework, called OMS, decoupling the UI from the Aggregation (Data) Layer.

\paragraph{Monitoring the Trigger System}

This work concerns one of the sources of monitored data: RateMon, the software providing Trigger Rates data, querying the OMDS database and carrying out \textbf{normalisations} and \textbf{corrections} for a number of different configurations and conditions, allowing consistent comparisons.

The Trigger system is responsible of filtering the large majority of events, spotting the potentially interesting ones, triggering the detector's read-out system to actually record data from the selected collissions. Monitoring the rates of such filters is essential to spot any anomalous behaviour in the underlying (sub)systems, software and/or hardware configurations, network and detector malfunctions.

Trigger Rates are presented in the form of \textit{Rates VS PU} \textbf{plots}. It also responsible of \textbf{alerting} the Trigger Shifters staff when recorded rates deviate too much from the \textbf{predicted} values. Those predictions are based on analytical models fitted on previously collected data.

\section{Scope}

TODO

\section{Structure}

We will start by introducing CERN, the research institute operating the LHC, outlining why such an experiment is needed and how it enables particle physics research.

Since this etc etc EXPERIMENTAL PART

\section{Conventions}

Implementation work on existing software have been done with Merge Requests. Each described task is accompanied with a corresponding sitography entry. FIXME this phrase is horrible

Code snippets (called "listings", from now on) demonstrating the execution of commands in a shell are noted with the \mcode{\$} character. Preceding the \mcode{\$} character you sometimes notice a string specifying the hostname of the machine (as specified in CERN OpenStack) or the general infrastructure/service in which that commands must run to have effect. E.g. \mcode{lxplus \$ command} shows the command execution on one of the machine from LXPLUS \cite{LXPLUSServiceITDepartment-2020-10-01}, the CERN service offering access to machines running Linux CERN CentOS 7. If there's no specification before \mcode{\$}, the environment is not relevant. With \mcode{P5} we refer to the machines installed in Point 5 of the LHC, in Cessy, home to the CMS detector.

The produced source code has been reported here only partially, focusing only on the relevant and meaningful parts. Refer to the git repositories for the complete copies. Some listings also have truncated outputs for the same reasons.