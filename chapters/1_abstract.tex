\phantomsection
\addcontentsline{toc}{chapter}{Abstract}

\begin{vplace}[0.7]
	\renewcommand*\abstractname{Abstract}

	\begin{abstract}

		\fontsize{11}{12}\selectfont\baselineskip=1.2em

		High Energy Physics experiments involve large amount of complex systems subject to anomalies and malfunctioning. They generate a lot of high dimensional and contextual data that must be monitored and analysed by a Data Quality Monitoring team to validate and deliver certified data for physics analysis.

		At the European Organization for Nuclear Research (CERN), in the Large Hadron Collider (LHC), the Compact Muon Solenoid (CMS) experiment generates events at a rate of 40 MHz, each one carrying payloads averaging 1.5 MB. It is the job of the Trigger System to discard the large majority of this data and retain the most interesting one.

		This crucial phase is sensible to malfunctions of many of the underlying parts, from sub detectors to the trigger algorithms configurations.

		Shifters in the CMS Control Room use a Rate Monitoring software to monitor Trigger Rates and spot potential problems by looking at reference fits and other detector diagnostic data.

		We proceed to improve and renovate this software, exploiting the new features to integrate the Trigger data with another dataset from the new CMS Run Registry, describing the status over time of every part of the detector, including the ones for which the failures are identifiable from the monitoring of the trigger rates.

		The resulting dataset can potentially aid novel Anomaly Detection approaches to this challenge.

		\normalsize

	\end{abstract}

\end{vplace}

\thispagestyle{empty}

\pagebreak