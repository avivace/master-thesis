\chapter{Anomaly Detection on Trigger Rates}
\label{dataset}
\section{Problem Statement}

\section{Data}

\subsection{Dataset A: Detector state over time}

This dataset describes the status of the CMS detector and its sub-systems over the progress of the Run, for every Run since the experiment is active.

The time progress of the Run is expressed in LS (Lumi Sections) units (\ref{ls_def}).

More than 25 thousands Run are available, starting from the 2009 collisions to the 2018 cosmics.

It's been generated using the CMS Run Registry API, with the DQM Offline Datasets as source.

\begin{listing}[H]
\begin{jsoncode}
  {
    "class": "Collisions18",
    "dataset_attributes": {},
    "datasets_in_gui": [],
    "deleted": false,
    "lumisections": {
      "btag-btag": {
        "EMPTY": 0,
        "GOOD": 228,
        "causes": [
          "UNDEF"
        ],
        "comments": []
      },
      "csc-csc": {
        "EMPTY": 0,
        "GOOD": 228,
        "causes": [
          "UNDEF"
        ],
        "comments": []
      },
      "dt-dt": {
        "EMPTY": 0,
        "GOOD": 228,
        "causes": [
          "UNDEF"
        ],
        "comments": []
      },
      "tracker-tracking": {
        "EMPTY": 0,
        "GOOD": 228,
        "causes": [
          "UNDEF"
        ],
        "comments": []
      },
      /* ... */
    },
    "name": "online",
    "run_number": 316361,
    "short_run": 1,
    "significant": true,
    "state": "SIGNOFF",
    "stop_reason": "ECAL preshower red recycle",
    "version": 4260755
  }
\end{jsoncode}
\caption{JSON export of the Run Registry data for Run 316361}
\end{listing}

\subsection{Dataset B: Trigger Rates}

For each Run we then exported the Trigger Rates in the form of "pre-deadtime unprescaled rate / num colliding bx" (Hz) values over LS time series.

There are hundreds of Triggers available, we selected the ones normally considered by shifters during Collision runs, which includes 17 HLT and 12 L1 Trigger Paths.

This data is generated querying data from the CMS OMDS database by the Rate Monitoring tools, which operates a series of normalisations to make the plots comparable across runs and configurations. A fitted function is also computed and included in the object describing the rates.

\begin{listing}[H]
\begin{jsoncode}
{
    "runnumber": 316361,
    "x_axis": "ls",
    "y_axis": "pre-dt-unprescaled-rate",
    "plots": {
        "HLT_DoubleEle33_CaloIdL_MW": {
            "plotname": "HLT_DoubleEle33_CaloIdL_MW_lumisection_vs_pre-deadtime unprescaled rate",
            "xvar": "lumisection",
            "yvar": "pre-deadtime unprescaled rate",
            "xVals": [
                23.0,
                24.0,
                25.0,
                // ...
            ],
            "yVals": [
                0.0038020953070372343,
                0.00429236376658082,
                0.005365777760744095,
                // ...
            ],
            "fit": {
                "linear": "0.00448 + x*-0.00000"
            }
        }
        // ...
    }
}

\end{jsoncode}
\caption{JSON export of Trigger Rates vs LS time series for Run 316361}
\end{listing}

\section{Anomaly Detection Pipeline}

\section{Results}