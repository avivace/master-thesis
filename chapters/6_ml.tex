\chapter{Machine Learning}

\section{Artificial Neural Networks}

\textit{Artificial Neural Networks} are mathematical models inspired by biological neural networks. They are composed of (layers of) connected neurons and characterized by processes changing their structure based on the flowing information during the \textit{learning} stage.

The first mention of this term is in McCulloch and Pitts' 1943 work \cite{McCulloch1943}, where they introduced a Threshold Logic Unit (or or Linear Threshold Unit) \ref{fig:TLU}, able to implement simple boolean functions.

\begin{figure}[H]
	\centering
	\begin{equation*}
		\begin{aligned}[c]
			TLU(\vec{x}) & = \phi \bigg( \sum_{i=1}^{n}{w_{i} \cdot x_{i}} \bigg) \\
			\phi(x)      & = x - \theta \geq 0
		\end{aligned}
		\quad
		\begin{aligned}[c]
			n      & = 2   \\
			\theta & = 1.5 \\
		\end{aligned}
		\quad
		\begin{aligned}[c]
			w_{1} & = 1.0 \\
			w_{2} & = 1.0
		\end{aligned}
	\end{equation*}
	\caption{Boolean logic AND implemented by a Threshold Logic Unit}
	\label{fig:TLU}
\end{figure}